你好,\LaTeX!这个仓库主要保存我的\LaTeX 模版,用于各种文档的书写。为了实现自由扩展的需求,一切格式上的改动都放在
cls文件中,并且所有实质性内容都放在src文件夹下,main.tex只用于整理,作为顶层。

为了测试参考文献格式是否正确,使用一篇稀疏运算加速的论文\citeup{zhang2020sparch}以及一篇老化预测的论文\citeup{sadi2017design}作为
参考文献样例。

深度神经网络已经发展成为解决传统机器学习任务最流行的技术,然而因为存在计算量大,参数量多的问题,将深度神经网络部署到硬件资源有限的嵌入式设备上非常困难。由于深度神经网络的规模随着应用场景的复杂化而增大,为了使其能部署到硬件资源有限的嵌入式设备上,网络压缩是一种有效的方法。在应用一个网络时需要先进行训练,得到合适的参数后将其部署到硬件设备上进行推理。训练是从大量数据中学习正确的参数的一个过程,而推理是使用训练好的参数对新的输入进行运算得到预测结果的过程。通常训练时需要大量的参数,但训练完成后便不需要如此多的参数,因此可以在训练完成后,部署到硬件设备前对网络进行压缩。网络压缩主要分为剪枝和数据量化,剪枝是指剪去网络中不重要的参数,数据量化是指减少表示权重的数据位宽。网络压缩后,不仅减小了进行推断所需的计算量,同时也减小了存储量和读写次数,如果压缩效果比较好,那么原本需要存储在DRAM中的数据就可以存储在SRAM中,进一步提高了访存效率。